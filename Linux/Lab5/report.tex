%% -*- coding: utf-8 -*-
\documentclass[12pt,a4paper]{scrartcl} 
\usepackage[utf8]{inputenc}
\usepackage[english,russian]{babel}
\usepackage{indentfirst}
\usepackage{misccorr}
\usepackage{graphicx}
\usepackage{amsmath}
\usepackage{hyperref}
\usepackage[nottoc,numbib]{tocbibind}
\linespread{1.3}

\begin{document}

\subsection*{Упражнение 1.2}
\noindent
Для заданных вещественных значений $a$, $b$, $c$, $d$, $e$, $f$, $x$ вычислить значение полинома $p(x)=ax^5+bx^4+cx^3+dx^2+ex+f$
\subsection*{Упражнение 2.2}
\noindent
Вычислить корни квадратного уравнения $ax^2+bx+c=0$ для любых вещественных значений $a$, $b$ и $c$. Напечатать значения корней, их количество и тип (вещественные, комплексные или мнимые).

\subsection*{Упражнение 3.1б}
\noindent
Для заданного массива $A$ из ста элементов вещественного типа вычислить сумму квадратов его элементов, начиная с первого.
  
\subsection*{Упражнение 4.1б}
\noindent
Протабулировать функции одной переменной:\\[2mm]
    $ f(x) = \ln x  +\sin^2 x,$\\
    $0.25 \leq x \leq 0.75,\triangle x = 0.05 $

\subsection*{Упражнение 5.2}
\noindent
В таблице из 1000 различных значений немонотонной функции $ f(x) $ найти и напечатать ее локальные минимумы (максимумы) и номера этих значений.

\subsection*{Упражнение 6.1б}
\noindent
Вычислить с заданной абсолютной погрежностью $ABSERR$ значения элементарных функций при заданном значении ар гумента $x$::\\[2mm]
    $$ cos x = 1 - \frac{x^2}{2!} + \frac{x^4}{4!} - \frac{x^6}{6!} + ...$$
    
\subsection*{Упражнение 7.1б}
\noindent
Упорядочить заданную числовую последовательность $a_1,a_2, ...,a_{100}$ так, чтобы:
$$a_i \leq a_{i+1}$$
\subsection*{Упражнение 7.7}
\noindent
Вычислить значение величины по заданным значениям трехмерной матрицы: $$q = \sum_{i=1}^{n} \prod_{j=1}^{m} \sum_{k=1}^{l}$$
\subsection*{Упражнение 7.17б}
\noindent
В массиве $A(100,50)$ найти элемент, являющийся наименьшим.    
\subsection*{Упражнение 7.27}
\noindent
В заданной матрице $B(30,40)$ поменять местами соседние столбцы (строки), т. е. первый со вторым, третий с четвертым, пятый с шестым и т. д.
\subsection*{Упражнение 7.37}
\noindent
Из элементов одномерного массива $A(100)$ сформировать трехмерный массив  $C(2,10,5)$ таким образом, чтобы при этом формировании быстрее менялся первый индекс массива C.

\subsection*{Упражнение 8.2}
\noindent
Составить процедуру умножения матрицы $B(M, N)$ на вектор $C(N)$. Применить ее для заданного вектора $A(10)$ и матрицы $D(15,10)$, элементы которой нао предварительно сформировать по правилу $d_{ij}=i(j+5)$.
\end{document}
